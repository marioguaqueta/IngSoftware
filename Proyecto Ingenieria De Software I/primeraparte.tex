\title{Sistema de Información para Negocio PegaMaz}
\author{Jorge Mario Guaqueta Restrepo}
\logo{img/logoUDistrital.png}
\legend{Proyecto del curso Ingeniería de Software I}
\director{Hector Florez Fernandez}
\institution{Universidad Distrital Francisco José de Caldas}
\degree{Ingeniería de Software I}
\faculty{Facultad de Ingeniería}
\city{Bogotá D.C.}
\date{Abril 1 de 2017}
\maketitle

\pagenumbering{roman}

\newpage
\phantomsection
\addcontentsline{toc}{chapter}{Resumen}
\chapter*{Resumen\markboth{Resumen}{}}
El proyecto consiste en un sistema de información en donde el cliente (\textbf{Pega Maz}) puede registar sus ventas de manera manual, posteriormente al momento de realizar el conteo de sus ventas este le arrojará un neto de las mismas asi llevando una contabilidad.
Tendra un modulo de contabilidad en el cual podra revisar las ventas dentro de unas determinadas fechas de filtro.

\tableofcontents
\listoffigures
\listoftables

\newpage
\pagenumbering{arabic}

\phantomsection
\addcontentsline{toc}{chapter}{Introducción}
\chapter*{Introducción\markboth{Introducción}{}}

Durante mucho tiempo muchas empresas o negocios han llevado su contabilidad de forma manual, asi como también sus inventarios y su facturación por lo que se hace necesario implementar sistemas de información en los cuales se pueda realizar una cierta cantidad de actividades que permiten un mejor manejo de los costos, gastos y flujos que se presenten en el negocio.

En cuanto a la necesidad se plantea un sistema de informacion el cual permitira facturar los productos que seran vendidos, ademas de permitir mediante el modulo de inventarios llevar un inventario de los productos que estan siendo comercializados y un modulo para obtener las ventas del dia, de la semana o del mes.


